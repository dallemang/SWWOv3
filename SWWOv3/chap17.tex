\chapter{Conclusions}
\label{ch17}



For those readers who are accustomed to various sorts of knowledge
modeling, the Semantic Web looks familiar. The notions of classes,
subclasses, properties, and instances have been the mainstay of
knowledge modeling and object systems modeling for decades. It is not
uncommon to hear a veteran of one of these technologies look at the
Semantic Web and mutter, ''Same old, same old'' indicating that there
is nothing new going on here and that everything in the Semantic Web has
already been done under some other name elsewhere.

As the old saying goes, ``There is nothing new under the sun,'' and to
the extent that the saying is correct, so are these folks when they
speak of the Semantic Web. The modeling structures we have examined in
this book do have a strong connection to a heritage of knowledge
modeling languages. But there is something new that has come along since
the early days of expert systems and object-oriented programming;
something that has had a far more revolutionizing effect on culture,
business, commerce, education, and society than any expert system
designer ever dreamed of. It is something so revolutionary that it is
often compared in cultural significance to the invention of the printing
press. That something new is the World Wide Web.

The Semantic Web is the application of advanced technologies that have
been used in the context of artificial intelligence, expert systems and
business rules execution in the context of a World Wide Web of
information. The Semantic Web is not simply an application running on
the Web somewhere; it is a part of the very infrastructure of the Web.
It isn't on the Web; it is the Web.

Why is this important? What is it that is so special about the Web? Why
has it been so successful, more so than just about any computer system
that has come before it?

In the early days of the commercial Web, there was a television ad for a
search engine. In the ad, a woman driving a stylish sports car is pulled
over by traffic policeman for speeding. As he prepares to cite her, she
outlines for him all the statistics about error rates in the various
machines used by traffic policemen for detecting speeding. He is clearly
thrown off his game and unsure of how to continue to cite her. She adds
personal insult by quoting the statistics of prolonged exposure to
traffic radar machines on sperm count. The slogan ``Knowledge is Power''
scrolls over the screen, along with the name of the search engine.

What lesson can we learn from ads like this? This kind of advertising
made a break from television advertising that had come before. Knowledge
was seen not as nerdy or academic but useful in everyday life---and even
sexy. Or at least it is if you have the right knowledge at the right
time. The Web differed from information systems that preceded it by
bringing information from many sources---indeed, sources from around the
world---to one's fingertips. In comparison to Hypercard stacks that had
been around for decades, the Web was an open system. Anyone in the world
could contribute, and everyone could benefit from that contribution.
Having all that information available was more important than how well a
small amount of information was organized.

The Semantic Web differs from expert systems in pretty much the same
way. Compared to the knowledge representations systems that were
developed in the context of expert systems, OWL is quite
primitive. But this is appropriate for a Web language. The power of the
Semantic Web comes from the Web aspect. Even a primitive knowledge
modeling language can yield impressive results when it uses information
from sources from around the world. In expert systems terms, the goals
of the Semantic Web are also modest. The idea of an expert system was
that it could behave in a problem-solving setting with a performance
that would qualify as expert-level if a human were to accomplish it.
What we learned from the World Wide Web (and the story of the woman
beating the speeding ticket) is that typically people don't want
machines to behave like experts; they want to have access to information
so they can exhibit expert performance at just the right time. As we saw
in the ad, the World Wide Web was successful early on in making this
happen, as long as someone is willing to read the relevant web pages,
digest the information, and sift out what he or she needs.

The Semantic Web takes this idea one step further. The Web is effective
at bringing any single resource to the attention of a Web user, but if
the information the user needs is not represented in a single place, the
job of integration rests with the user. The Semantic Web doesn't use
expert system technology to replicate the behavior of an expert; it uses
expert system technology to gather information so an individual can have
integrated access to the web of information.

Being part of the Web infrastructure is no simple matter. On the Web,
any reference is a global reference. The issue of managing global names
for anything we want to talk about is a fundamental Web issue, not just
a Semantic Web issue. The Semantic Web uses the notion of a URI as the
globally resolvable reference to a resource as a way of taking advantage
of the web infrastructure. Most programming and modeling languages have
a mechanism whereby names can be organized into spaces (so that you and
I can use the same name in different ways but still keep them straight
when our systems have to interface).

With the World Wide Web, the notion of a name in a namespace must be
global in the entire Web.
The URI is the Web-standard mechanism to do this; hence, the Semantic
Web uses the URI for global namespace identification. Using this
approach allows the Semantic Web to borrow the modularity of the World
Wide Web. Two models that were developed in isolation can be merged
simply by referring to resources in both of them in the same statement.
Since the names are always maintained as global identifiers, there is no
ad hoc need to integrate identifiers each time; the system for global
identity is part of the infrastructure.

An important contributor to the success of the World Wide Web is its
openness. Anyone can contribute to the body of information, including
people who, for one reason or another, might publish information that
someone else would consider misleading, objectionable, or just
incorrect. At first blush, a chaotic free-for-all of this sort seems
insane. How could it ever be useful? The success of the Web in general
(and information archiving sites like Wikipedia in particular) has shown
that there is sufficient incentive to publish quality data to make the
overall Web a useful and even essential structure.

This openness has serious ramifications in the Semantic Web, which go
beyond considerations that were important for technologies like expert
systems. One of the reasons why the Web was more successful than
Hypercard was because the Web infrastructure was resilient to missing or
broken links (the ``404 Error''). The Semantic Web must be resilient in
a similar way. Thus, inferencing in the Semantic Web must be done very
conservatively, according to the Open World assumption. At any time, new
information could become available that could undermine conclusions that
have already been made, and our inference policy must be robust in such
situations.

In the World Wide Web, the openness of the system presents a potential
problem. How does the heroine of the search engine commercial know that
the information she has found about radar-based speed detection devices is correct? She might have learned it from a
trusted source (say, a government study on these devices), or she might
have cross-referenced the information with other sources until she had
enough corroborating evidence to be certain. Or perhaps she doesn't
really care if it is correct but only that she can convince the traffic
cop that it is. Trust of information on the Web is done with a healthy
dose of skepticism but in the same way as trust in other media like
newspapers, books, and magazine articles.

In the case of the Semantic Web, trust issues are more subtle.
Information from the Semantic Web is
an amalgam of information from multiple sources. How do we judge our
trust in such a result even if we know about all the sources? To some
extent, the same principles apply. We can trust entities that we know or
have experience with, and we can trust entities that have gone through
some process of authorization and authentication. When we combine
information, we must also understand the impact that each information
source has on the outcome and what risk we are taking if we cannot trust
that source. These important issues for understanding the reliability of
the Semantic Web are still a subject of research.

In this book, we examined the modeling aspects of the Semantic Web: How
do you represent information in such a way that it is responsive to a
web environment? The basic principles underlying the Semantic Web---the
AAA slogan, the Nonunique Naming assumption, and the Open World
assumption---are constraints placed on a representation system if it
wants to function as the foundation of a World Wide Web of information.
These constraints have led to the main design decisions for the Semantic
Web languages of RDF, RDFS, and OWL.

There is more to a web than just the information and how it is modeled.
At some point, this information must be stored in a computer, accessed
by end users, and transmitted across an information network.
Furthermore, no triple store, and no inference engine, will ever be able
to scale to the size of the World Wide Semantic Web. This is clearly
impossible, since the Web itself grows continually. In the light of this
observation, how can the World Wide Semantic Web ever come to pass?

The applications we discussed in this book demonstrate how a modest
amount of information, represented flexibly so that it can be merged in
novel ways, provides a new dynamic for information distribution and
sharing. SKOS allows thesaurus managers around the globe to share,
connect, and compare terminology. QUDT aligns multiple applications so
that their measurable quantities can be combined and compared. OBO
Ontologies coordinate efforts of independent life sciences researchers
around the globe.

How is it possible to get the benefit of a global network of data if no
machine is powerful enough to
store, inference over, and query the whole network? As we have seen, it
isn't necessary that a Semantic Web application be able to access and
merge every page on the Web at once. The Semantic Web is useful as long
as an application can access and merge any web page. Since we can't hold
all the Semantic Web pages in one store at once, we have to proceed with
the understanding that there could always be more information that we
don't have access to at any one point. This is why the Open World
assumption is central to the infrastructure of the Semantic Web.

This book is about modeling in the context of the Semantic Web. What
role does a model play in the big vision? The World Wide Web that we see
every day is made up primarily of documents, which are read and digested
by people browsing the Web. But behind many of these web pages, there
are databases that contain far more information than is actually
displayed on a page. To make all this information available as a global,
integrated whole, we need a way to specify how information in one place
relates to information somewhere else. Models on the Semantic Web play
the role of the intermediaries that describe the relationships among
information from various sources.

Look at the cover of this book. An engineering handbook for aquifers
provides information about conduits, ducts, and channels sufficient to
inform an engineer about the pieces of a dynamic fluid system that can
control a series of waterways like these. The handbook won't give final
designs, but it will provide insight about how the pieces can be fit
together to accomplish certain engineering goals. A creative engineer
can use this information to construct a dynamic flow system for his own
needs.

So is the case with this book. The standard languages of RDF, RDFS, and
OWL provide the framework for the pieces an engineer can use to build a
model with dynamic behavior. Particular constructs like subClassOf and
subPropertyOf provide mechanisms for specifying how information flows
through the model. More advanced constructions like owl:Restriction
provide ways to specify complex relations between other parts of the
model. The examples from the ``in the wild'' chapters show how these
pieces have been assembled by working ontologists into complex dynamic
models that achieve particular goals. This is the craft of modeling in
the Semantic Web---combining the building blocks in useful ways to
create a dynamic system through which the data of the Semantic Web can
flow.
