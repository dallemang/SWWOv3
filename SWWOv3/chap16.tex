\chapter{Expert modeling in OWL}
\label{ch16}

The examples in Chapters\ref{ch10} and \ref{ch14} have shown applications of Semantic
Web, and the role that models play in those applications. In some cases
(e.g., OGP), a very simple model was used to mediate large amounts of
data. In other cases (e.g., QUDT and CHEBI), the models were very
involved, and a lot of the value came from the complexity of the model
itself. A large portion of applications on the Semantic Web can be
achieved using the modeling facilities that have been presented in the
book so far. But there are occasions where these modeling constructs are
insufficient, and more advanced capabilities are required. Many of these
have been included in version 2 of the OWL recommendation. We have already
seen (in Chapter\ref{ch13}) some of the advanced counting facilities from OWL
2, and we will outline some of the more fundamental new capabilities
here, but the OWL 2 recommendation is rich in modeling constructs that go
beyond the scope of this book.  Details of the OWL 2 recommendation can be found at
\cite{Parsia:12:OWO}. First and foremost, we will provide some
of the background you will need to search through the OWL recommendation
documents yourself to explore its rich landscape.

OWL 2 was designed to be fully backward compatible with version 1.0.
That means that any valid OWL 1.0 model is also a valid model in OWL 2.
But more importantly, it means that all the styles of modeling that we
learned for OWL 1.0 are still valid for OWL 2. The new constructs in OWL
2 can be used in conjunction with all the models you've seen in the book
so far.

In Chapter~\ref{ch9}, we introduced a subset of OWL that we called RDFS-Plus.
There are a number of reasons why someone might define a subset of a
language like OWL. In the case of RDFS-Plus, we were interested in a
subset of the language that has considerable utility for semantic
modeling but does not place a large burden on either a modeler or
someone trying to understand a model. OWL 2 also includes a precise
description of four subsets of the OWL language (each of them richer
than RDFS-Plus) identified for various practical technological reasons,
often having to do with how OWL relates to other technology. The four
subsets are called OWL 2 EL, OWL 2 QL, OWL 2 RL, and OWL 2 DL.


We will describe each subset and the rationale for why it has been
identified and named; the details of each subset are given in the OWL
specification.

\section{OWL Subsets and Modeling Philosophy}

Normally, when we refer to different subsets of a language, we can list
the language structures in one subset that are not found in the other.
For instance, RDFS has \texttt{rdfs:domain}, \texttt{rdfs:range}, \texttt{rdfs:subPropertyOf}, and
so on, whereas RDFS-Plus has all of those, plus some new language
features like \texttt{owl:inverseOf} and \texttt{owl:TransitiveProperty}. We can define
how these two languages are similar or different, based on which
language terms are available in each one.

In the case of the OWL 2 subsets, the situation is more subtle. Each
subset uses the same set of modeling constructs. That is, if we were to
list all the properties and classes that make up (say) OWL EL and then
compile the full list for all of OWL, the lists would be exactly the
same. In fact, it would be the list of OWL features that you have been
reading about in this book. Everything you have learned so far applies
equally well to each OWL subset.

So what is the difference? Why is it important to identify subsets of a
language, if they have the same constructs and the same meanings? The
distinction between these the subsets of OWL are motivated in part by a
difference in the basic philosophy of why one builds models for the
Semantic Web. We will outline these two basic philosophies---one in
which the emphasis is placed on having provable models and one in which
the emphasis is placed on making executable models. We examine each in
turn, along with the intuitions that motivate them.

\subsection{Provable models}

An important motivation for formal modeling (as opposed to informal
modeling) is to be precise about what our models mean. In the context of
the Semantic Web, this tells us precisely and without doubt when
concepts from two different sources refer to the same thing. Does my
notion of a James Dean movie correspond to yours? A formal description
can help us determine whether or not this is the case. My definition of
a ``James Dean movie'' is one that stars James Dean, but your definition
of a ``James Dean movie'' might be movies about James Dean or movies
with the words James Dean in the title. How can we tell if we have the
name ``James Dean movie'' as the only indication of these definitions? A
formal model makes these distinctions clearer. Then it becomes a simple
matter of automation to decide whether two classes are the same, if one
subsumes the other, or if they are unrelated.

It is this aspect of modeling that motivates a logical definition of
OWL. Each construct in OWL is a statement in a formal logic. The
particular logical system of OWL DL is called Description Logic. As the
name suggests, Description Logic is a logical system with which formal
descriptions of classes, individuals, and the relationships between them
can be made. The inferences in OWL that have formed the basis of the
bulk of this book are formally defined by a model theory based on
Description Logic.

Using logic as the foundation of a modeling language makes perfect
sense; we can draw upon decades, or even centuries, of development work
in logical formalism. The properties of various
logical structures are well understood. Logic provides a framework for
defining all of the inferences that our modeling language will need. But
there is one fly in the ointment: In a computational setting, we would
like our logic to be processed automatically by a computer.
Specifically, we want a computer to be able to determine all of the
inferences that any given model entails. So, if we want to be able to
automatically determine whether my notion of a James Dean movie is
exactly the same as yours, we must show the set of all facts that are
true in one are true in the others, and all facts untrue are untrue.

It is at this point that the details of the logic become important. What
does it mean for our modeling
formalism if we base it on a logic for which this kind of automation
cannot, in principle, exist? That is, what happens if we can't exactly
determine whether my notion of a James Dean movie is the same as yours?
If we view this sort of provable connection as essential to the nature
of modeling, then we have failed. We simply cannot tolerate a logic in
which this kind of question cannot be answered by automated means in
some finite amount of time.

In the study of formal logic, this question is called \emph{decidability}.
Formally, a system is decidable if there exists an effective method such
that for every formula in the system the method is capable of deciding
whether the formula is valid (is a theorem) in the system or not. If
not, then the system is undecidable. It is not our intention in this
book to go into any detail about the mathematical notion of
decidability, but a few comments on its relevance for modeling are in
order.

The first thing to understand about decidability is also the most
surprising: how easy it is for a formal system to be undecidable. Given
the formal nature of logic, it might seem that, with enough patience and
engineering, a program could be developed to correctly and completely
process any formal logic. One of the most influential theorems that
established the importance of the notion of decidability shows that even
very simple logical systems (basically, any system that can do ordinary
integer arithmetic) are undecidable. In fact, it is actually quite
challenging to come up with a logical system that can represent anything
useful that is also decidable.

This bit of tightrope walking is the impetus behind the OWL DL subset.
OWL DL is based on a particular decidable \emph{Description Logic}. This means
that it is possible to design an algorithm that can take as input any
model expressed in OWL DL and determine which classes are equivalent to
other classes, which classes are subclasses of other classes, and which
individuals are members of which classes.

The most commonly used algorithm for this problem is called the \emph{Tableau
Algorithm}. It works basically by keeping track of all the possible
relations between classes, ruling out those that are inconsistent with
the logical statements made in the model. The Tableau Algorithm is
guaranteed to find all entailments of a model in OWL DL in a finite (but
possibly quite long!) time. Furthermore, it is possible to determine
automatically whether a model is in fact in OWL DL so that a program can
even signal when the guarantees cannot be met.

Modeling in OWL DL supports the intuition that a model must be clear,
unambiguous, and machine-processable. The Tableau Algorithm provides the
machinery by which a computer system can make determinations about
equivalence of classes.

\subsection{Executable models}

A different motivation for modeling in the Semantic Web is to form an
integrated picture of some sort of domain by federating information from
multiple sources. If one source provides information about the places
where hotel chains have hotels and another describes what hotels appear
at a particular
place, a formal model can tell us that we can merge these two sources
together by treating them as inverses of one another. The model provides
a recipe for adding new information to incomplete information so it can
be federated with other sources.

Seen from this point of view, a model is similar to a program. It
provides a concise description of how data can be transformed for use in
other situations. What is the impact of decidability in such a
situation? Standard programming languages like FORTRAN and Java are
undecidable in this sense. The undecidability of these languages is
often demonstrated with reference to the \emph{Halting Problem}. It is
impossible in principle to write a computer program that can take
another arbitrary computer program as input, along with input for that
program, and determine whether that program will halt on that input.
Even though these languages are undecidable, they have proven
nevertheless to be useful engineering languages. How can we write
programs in these languages if we can't automatically determine their
correctness or, in some sense, even their meaning? The answer to this
question in these cases is what programming is all about. Even though it
is not possible in general to determine whether any program will
terminate, it is usually possible to determine that some particular
program will terminate and, indeed, with what answer. The skill of
engineering good computer programs is to write programs that not only
will terminate on all input but will actually perform well on
particularly interesting input.

Seen from this point of view, decidability is not a primary concern.
Models are engineered in much the same way as programs. If a model
behaves poorly in some situation, then an engineer debugs the model
until it performs correctly. Since we are not concerned with
decidability, we don't need the guarantee that any algorithm will find
all possible inferences. This opens up the choice of processor for OWL
to a much wider range of technologies, including rule systems, datalog
engines, databases, and even SPARQL.

It's also the case that, in many Web applications, the size of data sets
we would like to analyze are quite huge, dynamic, or not well organized.
The question could be asked as to whether one needs a 100 percent
correct model to analyze data that are themselves scraped from the Web
by some heuristic program that is not perfect. On the Web, people use
Google because it can find good answers a lot of the time, even if it
can't find perfect answers all the time. Some Semantic Web systems are
targeted at this rough-and-tumble Web application space, and thus
provable correctness, as opposed to efficient computation, may not be a
key goal.

This \emph{executable} style of modeling is the primary motivation behind some
of the OWL subsets.

The meaning of a modeling construct in one of these subsets is given in
much the same way as the meaning of a construct in a programming
language. Just as the meaning of a statement in a procedural programming
language is given by the operation(s) that a machine will carry out when
executing that statement, the meaning of an executable model is given by
the operation(s) that a program (i.e., an inference engine) carries out
when processing the model. Information federation is accomplished
because the model describes how information can be transformed into a
uniform structure.

\section{OWL 2 Modeling Capabilities}

A comprehensive list of the advanced modeling capabilities supported by
OWL 2 is beyond the scope of this book, but we describe some of the most
important ones here.

\subsection{Metamodeling}

Metamodeling is the name commonly given to the practice of using a model
to describe another model as an instance. One feature of metamodeling is
that it must be possible to assign properties to classes in the model.
This practice causes a problem in OWL 1.0, since OWL 1.0 disallowed
treating classes as if they were individuals in this way.  Metamodeling is one of the 
motivation for \emph{Punning}, as described (as an anti-pattern) in Section~\ref{punning}.

One motivation for metamodeling is that a model often needs to play more
than one role in an application: A particular concept should be viewed
as a class in one role but as an instance in another role. If we are
modeling animals, we might say that \texttt{BaldEagle} is an endangered species,
thereby referencing \texttt{BaldEagle} as an individual. In another application,
we could view \texttt{BaldEagle} as a class, whose members are the particular
eagles in the zoo. Similarly, wine connoisseurs speak of individual
wines in terms of vintage. For them, the vintage is an individual, but
for a wine merchant who is calculating how many bottles he has sold, the
bottles themselves are individual members of the class that are
indicated by the vintage.

We have already seen some example of this kind of metamodeling in this
book. In Chapter~\ref{ch10}, we saw how a \texttt{foaf:Group} is an individual that
corresponds to a class of all the members of the group. In Chapter~\ref{ch12},
we saw this in the \emph{Class-Individual Mirror} pattern.

Another purpose of metamodeling is to imitate capabilities of other
modeling systems (like object-oriented modeling) in which the value for
some property can be specified for all members of a class at once. The
overloading of a resource to refer both to an individual (the species
\texttt{BaldEagle}) and a class (the set \texttt{BaldEagle}) is allowed in OWL 2 through a
process known as \emph{punning}.

While punning is allowed in the OWL 2 recommendation, we recommend against its
use for metamodeling.

There really is a difference between a species and the set of animals of
that species; there is a difference between the desktop and the
applications that run on it. The relationship between a bottle of wine
and its vintage is different from the relationship between an eagle and
its species, and these distinctions could be important to someone who
wants to reuse a model. Keeping them distinct in the first place will
often enhance the model's utility.

In particular, we recommend the use of the \emph{Class-Individual Mirror}
pattern from Chapter~\ref{ch12} for metamodeling. In that example, the
relationship between the desktop and the application was clear (\texttt{runsOn}),
as was the relationship between the ADA90 and the applications
(\texttt{conformsTo}). When modeling a class like \texttt{BaldEagle}, we recommend
determining just what the relationship is between a particular eagle and
the class (\texttt{hasSpecies}), or the particular bottle and the vintage
(\texttt{hasVintage}). Just as in our example of the desktop applications
conforming to the ADA90, this allows the model to relate these
properties to others---for example, modeling \texttt{hasSpecies} explicitly
allows the model to relate it to other properties like \texttt{hasGenus} or
\texttt{hasPhylum}. Modeling \texttt{hasVintage} explicitly allows the model to express
relationships to other properties like \texttt{madeInYear}.

\subsection{Multipart properties}

In RDFS, we have seen how properties can relate to one another using
\texttt{rdfs:subPropertyOf}. This establishes a hierarchy of properties: Any
relations that hold lower in the hierarchy also hold higher in the
hierarchy. There are other ways in which properties can relate to one
another. A common example is the notion of uncle: A is the uncle of B
only if A is the brother of someone who is the parent
of B. This is called a multipart property---that is, the property uncle
is made up of two parts (in order): parent and brother. We have already
seen how to define relationships of this sort using SPARQL in Chapter~\ref{ch5}.

When multipart properties are used with other RDFS and OWL constructs,
they provide some powerful modeling facilities. For instance, we can
model the constraint ``A child should have the same species as its
parent'' by stating that the multipart predicate made up of \texttt{hasParent}
followed by \texttt{hasSpecies} (denoted as \texttt{:hasParent} $+$ \texttt{:hasSpecies}) is
\texttt{rdfs:subPropertyOf} \texttt{hasSpecies}. Let's see how this works. Suppose we have
the following triples:

\begin{lstlisting}
:Elsie :hasParent :Lulu . 
:Lulu :hasSpecies :Cow .
\end{lstlisting}

Now we can infer

\begin{lstlisting}
:Elsie :hasParent + :hasSpecies :Cow .
\end{lstlisting}

But since the multipart predicate \texttt{:hasParent} + \texttt{:hasSpecies} is an
\texttt{rdfs:subPropertyOf}
\texttt{:hasSpecies}, we can infer that

\begin{lstlisting}
:Elsie :hasSpecies :Cow.
\end{lstlisting}

One reason that multipart predicates were not included in OWL 1.0 was
that they were thought to cause undecidability. Recently, however, it
has been shown that under certain conditions it is possible to represent
multipart properties in OWL in such a way that they do not endanger
decidability. The multipart predicate feature in OWL 2.0 has been
designed to guarantee decidability.

\subsection{Multiple inverse functional properties}
\label{mIFP}
Inverse functional properties can be used to determine the identity of
individuals based on the values of the properties that describe them. If
two people share the same social security number, then we can infer that
they are actually the same person. This kind of unique identifier is
indispensable when merging information from multiple sources.

Unfortunately, anyone who has done a lot of such integration knows that
this kind of merging only scrapes the surface of what needs to be done.
Far more common is the situation in which some combination of properties
implies the identity of two or more individuals. For instance, two
people residing at the same residence with the same first and last names
should be considered to be the same person. Two people born in the same
hospital on the same day and at the same time of day should be
considered to be the same person. Examples of this kind of multiple
identifiers are much easier to come by than single identifiers, as
required for an InverseFunctionalProperty.

OWL 2 introduces the notion of \texttt{owl:hasKey} for this situation. By analogy
to how a relational
database can declare multiple primary keys for a table, a set of
properties can be associated with a class through \texttt{owl:hasKey}. Two
members of the class are considered to be the same (\texttt{owl:sameAs}) if they
have the same values for all the identified keys. If we were to define
keys \texttt{:firstName},
\texttt{:lastName}, and \texttt{:address} for the class \texttt{:Person}, then two people would be
considered the same
whenever all of these properties match.

To further complicate matters, in many information federation
situations, it is often the case that even these combinations of
properties cannot guarantee the identity of the individuals. Two people
at the same address with the same name are very likely to be the same
person (but not for certain--- a father could live with his son of the
same name). (Several schemes for adding uncertainty to OWL have been proposed, but nothing has been standardized to date, so there is no
way to express this sort of information.

\subsection{OWL 2 profiles}

OWL 2 includes one major infrastructural change from OWL 1.0 in the
introduction of several profiles of the language. The profiles were
motivated by the observation that certain technologies can process
certain subsets of OWL conveniently. If one has already made a
commitment to such a technology, it is natural to ask just what subset
is supported by that technology.

There are four subsets of OWL that have been named and identified.

\begin{itemize}
\item \textbf{OWL 2 DL}. Decidability can be a key driver for model development. In
such circumstances, it is desirable to have as expressive a language as
possible, while still being decidable. OWL 2 DL is the largest subset of
OWL that retains this feature. It includes all forms of restrictions,
combined with all of the RDFS forms. It is a superset of the next three
profiles. It can be processed faithfully by the tableau algorithm.

\item \textbf{OWL 2 QL}. Many semantic applications leverage information in relational
databases, and need to be built on top of these systems. Such
applications fit the profile of typical database applications, in which
a fairly simple schema describes the structure of massive amounts of
data, and fast responses to queries over that data set are required. OWL
2 QL is the subset of OWL for which this is possible. Queries against an
OWL 2 QL ontology and corresponding data can be rewritten faithfully
into SQL, the query language of relational databases.

\item \textbf{OWL 2 EL}. Many ontologies in the life sciences follow the pattern of the
OBO Foundry ontologies described in Chapter 14; they include a large
number of classes that are defined primarily using someValuesFrom
restrictions. It is difficult to process ontologies of this size using
an unconstrained tableau algorithm. The OWL 2 EL profile was designed
for just this case. It walks the line between expressive power and known
optimizations for querying such structures. OWL 2 EL allows
someValuesFrom restrictions to be used, so the OBO Foundry ontologies
fit within its limitations. But it is restricted enough that fast
algorithms are known for processing large ontologies, up to and
surpassing the size of the OBO Foundry ontologies.

\item \textbf{OWL 2 RL}. Many OWL processors work by using rules-based technology to
define OWL processing. In this book, we have often used SPARQL to
illustrate the inference rules that define certain constructs in OWL and
RDFS. Used in this way, SPARQL is an example of such a rule processor.
Rules processors have been around about as long as relational databases,
in the form of systems like Prolog and Business Rules engines. OWL 2 RL
defines the subset of OWL that can be faithfully processed by such rule
systems. Having identified this subset, it is possible to encode the
rules for OWL 2 RL in each rule system. This exercise has already been
done for the W3C Rules Interchange Format (RIF \cite{Kifer:13:ROE}), as well as
many proprietary rules processors. Many variants of encodings of OWL 2
RL into SPARQL have been done as well.
\end{itemize}



It is important to keep in mind that all of the OWL subsets use the very
same resources in their models---there is no separate namespace for any
profile. That means that any model in any profile can be interpreted as
a model in any other---subject to the restrictions of that profile. In
this way, all the profiles are interoperable at the RDF level.

\subsection{Rules}

Even with the capabilities added in OWL 2, there are still some limits
to the expressivity of OWL. Some of these limitations are best
addressed, for the purposes of data management, using rules, and thus
the development of a rules language for the Web has been developed in
the form of the Rules Interchange Format (RIF).

Rule-based systems have a venerable tradition starting in the days of
Expert Systems and are in common use in business logic applications to
this day. A number of useful algorithms for processing data with rules
have been known for many years, and many of them have been made very
efficient.

Many of the issues with OWL presented in this chapter can be addressed
with rules. Multipart
properties (like the definition of uncle) are easily expressed in rules.
Multiple inverse functional properties can be expressed in rules as
well. There are even a number of approaches to reasoning with
uncertainty in rules. Many of these have considerable research and
practical examples behind them, making uncertainty in rules a relatively
well-understood issue.

Given all these virtues of rules and rule-based systems, why don't they
play a bigger role in modeling on the Semantic Web than they do? In
fact, one could even ask why there is a need for a modeling language
like OWL when there is a mature, well-understood rules technology that
already exists. One could even ask this question in greater generality.
Why aren't more software systems in general written in rules?

We cannot treat this issue in full detail in this book, but we can
outline the answer as it relates to OWL and the Semantic Web. One of the
lessons learned from the history of rule-based systems is that software
engineering in such systems is more difficult than it is in modular,
procedural languages. Although it is unclear whether or not this is an
essential feature of rule-based systems, it is undeniable that
rule-based programmers have not achieved the levels of productivity of
their more conventional counterparts. This has particular ramifications
in the Semantic Web. One defense for using an OWL subset other than OWL
2 DL was that the software engineering discipline makes the notion of
decidability basically irrelevant for model design. In the case of
rule-based systems, software engineering cannot provide this same
support. Unconstrained rule-based systems are just as undecidable as
general-purpose languages like FORTRAN and Java.

Is there a way to get the best of both worlds? Could a Web-oriented
rules language integrate well with OWL? Indeed they can, and this is
exactly what we are seeing in the development of things like OWL 2 RL
and RIF. Together, they provide a framework that is consistent with OWL,
as well as with broader rule-based technology.

\section{SUMMARY}

OWL should be considered a living language, growing in the context of
the ways it is being used on the

Web and in commerce. As shortcomings in the language are identified, the
system grows to
accommodate them. Sometimes that growth takes the form of additional
constructs in the language (e.g., multipart properties), sometimes as
connections to other systems (rules), and sometimes progress in a
language comes in the form of specifying subsets of the language.

\subsection{Fundamental concepts}

The following fundamental concepts were introduced in this chapter.

OWL---Web Ontology Language, including all constructs in this book and
more.

OWL 2 DL---Subset of OWL restricted to ensure decidability; all
constructs allowed but with certain restrictions on their use.

OWL 2 EL---Subset of OWL restricted to improve computational complexity.

OWL 2 RL---Subset of OWL restricted to be compatible with Rules
processors.

OWL 2 QL---Subset of OWL restricted to be compatible with database
queries.

RIF---The Rule Interchange Format, standard format in the Semantic Web
for interoperability of rule-based systems.

Metamodeling---Models that describe models, usually requires that
classes be treated as individuals.

Multipart properties---Daisy-chain composition of properties.

Multiple Inverse Functional Properties---Uniquely identify an individual
based on matching values for several properties.
